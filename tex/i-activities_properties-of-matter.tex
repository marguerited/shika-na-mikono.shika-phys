\section{Structure and Properties of Matter}

%States of Matter


%Particles / Kinetic Theory of Matter


%Elasticity


%Adhesion and Cohesion


\subsection{Determining Adhesion and Cohesion}

\subsubsection*{Learning Objectives}
\begin{itemize}
\item{To observe cohesion and adhesion forces of different liquids.}
\item{To determine adhesive forces between water molecules.}
\item{To determine cohesive forces between different liquids.}
\end{itemize}

\subsubsection*{Background Information}
Adhesion is the force of attraction between a material and its surroundings. Cohesion is the force of the liquid molecules to stick to themselves. The concept of adhesive and cohesive forces can be used in different ways such as determination of viscosity of the liquids and explaining the transportation of liquid in plants and animals. When a drop of water is placed on a sheet of glass, the water spreads because adhesive forces between glass and water are greater than the cohesive forces between water molecules. When a drop of honey, cooking oil, or glycerine is placed on a sheet of glass, it remains almost spherical because the cohesive force is greater than the adhesive force.

\subsubsection*{Materials}
Sheet of glass, water, honey, glycerine, cooking oil, syringe, and 2 wooden blocks

\subsubsection*{Activity Procedure}
\begin{enumerate}
\item{Place 2 wooden blocks on the top of the table.}
\item{Place a piece of pane glass horizontally over the two wooden blocks.}
\item{Using a syringe, place different drop of liquids on top glass. Record your observations.}
\end{enumerate}

\begin{figure}[h]
\begin{center}
\def\svgwidth{200pt}
\input{./img/adhesion-cohesion.pdf_tex}
\caption{Observing adhesive and cohesive forces}
\label{fig:adhesion-cohesion}
\end{center}
\end{figure}

\subsubsection*{Results and Conclusions}
When a drop of water is placed on top of glass, water spreads and wets the glass. While material like honey, glycerine, and cooking oil remain in a spherical shape. The adhesive forces between the water molecule and glass molecule is greater. While the cohesive forces between the molecule of honey, glycerine and cooking oil is larger.

\subsubsection*{Clean Up Procedure}
Collect all the used materials, cleaning and storing items that will be used later.

\subsubsection*{Discussion Questions}
\begin{enumerate}
\item{Explain why water wets the glass while glycerine does not wet the glass.}
\item{Discuss where we apply the concept of adhesive and cohesive forces in other areas of science.}
\item{How are cohesion and viscosity related?}
\end{enumerate}

\subsubsection*{Notes}
The liquids which have the greatest forces of cohesion are also those with the highest viscosity.


%\subsection{Pinching Water}
%\begin{itemize}
%\item{Preparation Time: 10 minutes}
%\item{Materials: 0.5 liter bottle, water, pin or small nail}
%\item{Procedure: At the bottom of the side of the can or bottle, poke five small holes close together with the syringe needle or nail. Be careful not to let the holes overlap or be too far apart. Pour water into the bottle and allow the water to start flowing out of the holes at the bottom. Using your thumb and forefinger, pinch the streams of water together so that they form a single stream (this takes some practice). To undo your great work of creation, pass your hand in front of the holes and five streams will appear again.}
%\item{Theory: Water has a tendency to “cling” to itself due to its surface tension and cohesion. As you bring the streams together, you allow the water to stick to itself forming a single stream. Passing your hand in front again stops the flow of water at the holes and allows it to start again, which it will do in five streams.}
%\end{itemize}


\subsection{Cohesion in a Moving Liquid}

\subsubsection*{Learning Objectives}
\begin{itemize}
\item{To observe the effect of cohesion on moving water}
\end{itemize}

\subsubsection*{Background Information}
Cohesion is the force between molecules in a liquid. It holds liquids like water together if they are standing still or moving.

\subsubsection*{Materials}
Empty 0.5 litre bottle, water, syringe needle OR pin OR small nail

\subsubsection*{Preparation Procedure}
Make five small holes at the bottom of the side of the bottle. Make the holes close together (about 5 mm apart) with the syringe needle or nail.

\subsubsection*{Activity Procedure}
\begin{enumerate}
\item{Pour water into the bottle and allow the water to start flowing out of the holes at the bottom.}
\item{Using your thumb and forefinger, pinch the streams of water together so that they form a single stream.}
\item{To separate the streams again, pass your hand in front of the holes and the five streams will appear again.}
\end{enumerate}

\subsubsection*{Results and Conclusions}
When you pinch the streams of water together, they form a single stream or a few streams. Water has a tendency to cling to itself due to its surface tension and cohesion. As you bring the streams together, you allow the water to stick to itself forming a single stream. Passing your hand in front again stops the flow of water at the holes and allows it to start again, which it will do in five streams.

\subsubsection*{Clean Up Procedure}
Return all materials to their proper places.

\subsubsection*{Discussion Questions}
\begin{itemize}
\item{How did the behaviour of the water streams change as the level in the bottle decreased?}
\item{What force holds the water streams together?}
\item{Why do the streams eventually split?}
\end{itemize}

\subsubsection*{Notes}
Be careful not to let the holes overlap or be too far apart. This will cause the water to form one stream from the beginning or make it impossible to pinch the streams. Practice pinching the water and make new holes if necessary.


%Surface Tension


\subsection{Water drops}
\begin{itemize}
\item{Preparation time: none}
\item{Materials: Water dropper or syringe}
\item{Procedure: Slowly drip water from the water dropper or syringe and point out that before a drop falls; it will hang suspended by its surface tension. Explain that as the drop becomes larger, its weight increases until surface tension is insufficient to support it, at which point it falls.}
\end{itemize}

%\subsection{Blowing bubbles}
%\begin{itemize}
%\item{Preparation time: 5 minutes}
%\item{Materials: Thin piece of wire (approximately 30cm), water, detergent, glycerin (optional)}
%\item{Procedure: Bend the wire into a loop 2 to 3 cm in diameter. Continue to bend the wire so that it circles around the circumference of this circle several times. Leave a straight piece several centimeters long to use as a handle. This is the “bubble blower”. Dip the circular part of the bubble blower into a strong solution of detergent (regular powdered laundry detergent works well) mixed with glycerin. When you remove the bubble blower from the solution, a thin film should remain across the circle. Gently blow through the center of the circle. With a little practice, you should be able to cause a spherical bubble to separate from the blower and float away.}
%\item{Theory: The detergent causes the surface tension in the solution to be slightly variable. In areas of higher concentration of detergent, the surface tension is lower. In order for the films and bubbles to be stable, the surface tension near the top must be slightly higher than at the bottom. As the detergent molecules are heavier than water, they tend to sink towards the bottom of the film, accomplishing this.\\
%When you blow through the bubble blower, we can see that then tension is pulling it back towards a flat surface. Once an independent bubble is formed, we see that it forms a nearly perfect sphere. This is because the surface is under tension. This tension forces the bubble to form the shape with the minimum surface area, a sphere. It is also worth noting that both the film that stays on the bubble blower and the bubbles themselves appear to have small rainbows of colors in them. This is caused by thin-film interference.}
%\end{itemize}


\subsection{Surface Tension in Bubbles}

\subsubsection*{Learning Objectives}
\begin{itemize}
\item{To observe the effect of surface tension in soapy water}
\end{itemize}

\subsubsection*{Background}
Surface tension is the force which holds the molecules together on the surface of a liquid. It can be strong enough that insects can walk on water, or objects with high density can float on a liquid. Bubbles form under the force of surface tension. As the surface of a liquid naturally forms a shape with the smallest surface area, a thin film of liquid will form a sphere, held together by surface tension around air. The air inside pushes out with equal force that air outside pushes in. Bubbles do not form easily in water, so a soapy solution is needed.

\subsubsection*{Materials}
Thin piece of wire (approximately 30cm), water, detergent, glycerin (optional)

\subsubsection*{Preparation Procedure}
\begin{enumerate}
\item{Bend the wire to form a loop of 2 to 3 cm in diameter.}
\item{Continue to bend the wire so that it circles around the circumference of this circle several times.}
\item{Leave a straight piece several centimeters long to use as a handle. This is the bubble blower.}
\item{prepare a concentrated solution of detergent by mixing powdered soap in water with a small amount of glycerin.}
\end{enumerate}

\subsubsection*{Activity Procedure}
\begin{enumerate}
\item{Dip the circular part of the bubble blower into the detergent.}
\item{When you remove the bubble blower from the solution, a thin soapy film should remain across the circle.}
\item{Gently blow through the center of the circle. With a little practice, you should be able to cause a spherical bubble to separate from the blower and float away.}
\end{enumerate}

\subsubsection*{Results and Conclusions}
When you blow through the bubble blower, we can see that the tension is pulling it back towards the surface. Once an independent bubble is formed, we see that it forms a nearly perfect sphere. This is because the surface is under tension. This tension forces the bubble to form the shape with the minimum surface area, a sphere.

\subsubsection*{Clean Up Procedure}
Return all materials to their proper places.

\subsubsection*{Discussion Questions}
\begin{enumerate}
\item{What is inside the bubble?}
\item{Compare the pressure inside the bubble to the pressure outside the bubble.}
\item{Why do bubbles form easily in soapy water but not in fresh water?}
\end{enumerate}

\subsubsection*{Notes}
Try making the solution without glycerin and note any different results.\\
It will be seen that the bubbles have colours; this is due to thin film interference, or the refraction of white light in a thin liquid medium.


\subsection{Pin Float}
\begin{itemize}
\item{Preparation time: none}
\item{Materials: A cup or small dish, a straight pin, water, detergent}
\item{Procedure: Make sure the cup or dish is clean, and has no soap or detergent residue. Fill the cup or dish with clean water. Carefully place the straight pin on the surface of the water, being careful not to break the surface. If done properly, it should be possible to get the straight pin to remain suspended on the surface (see also floating compass). Next, sprinkle a small amount of detergent onto the water. The pin should sink to the bottom.}
\item{Theory: When the straight pin is placed on the surface, it causes the surface of the water to bend downwards. This means that the surface tension of the water is pulling the pin upwards. Although the pin is denser than water, and would normally sink, this surface tension is enough to support the weight of the pin. When detergent is sprinkled onto the surface of the water, it lowers the surface tension of the water. The surface tension is no longer strong enough to hold up the pin, so the pin sinks.}
\end{itemize}

\subsection{Pepper Float}
\begin{itemize}
\item{Preparation time: none}
\item{Materials: A cup or dish, water, ground black pepper, soap or detergent}
\item{Procedure: Make sure that the cup or dish is clean, and has no soap or detergent residue. Fill the cup or dish with clean water. Sprinkle ground black pepper over the surface of the water in a way that the pepper is distributed evenly and covers the whole surface. Next, apply a small amount of soap or detergent to one finger. Touch this finger to the surface of the water in the center of the cup or dish. The pepper will flee your finger, and all run to the sides of the cup or dish.}
\item{Theory: When you touch your finger to the surface, you introduce a small amount of soap or detergent, lowering the surface tension at that point. The surface of the water is now unbalanced – the surface tension near the edge is pulling the surface outwards more strongly than the surface tension at the center is pulling the surface inwards. As there is a net force on the surface outwards towards the edge, the surface moves, pulling the pepper along with it to the edges of the cup or dish.}
\end{itemize}

%\subsection{Changing Surface Tension}
%
%\subsubsection*{Learning Objectives}
%\begin{itemize}
%\item{To observe the effect of surface tension}
%\item{To observe the effect of an impurity on the surface tension of water}
%\end{itemize}
%
%\subsubsection*{Background Information}
%Surface tension holds the molecules of a liquid together at the surface. However, the surface tension is not uniform; it depends on the composition of the liquid as well as the other forces present. In this experiment, we observe the effects of changing the surface tension.
%
%\subsubsection*{Materials}
%Cup or dish, water, ground black pepper, soap or detergent
%
%\subsubsection*{Preparation Procedure}
%Make sure that the cup or dish is clean, and has no soap or detergent residue.
%
%\subsubsection*{Activity Procedure}
%\begin{enumerate}
%\item{Fill the cup or dish with clean water.}
%\item{Sprinkle ground black pepper over the surface of the water in a way that the pepper is distributed evenly and covers the whole surface.}
%\item{Apply a small amount of soap or detergent to one finger.}
%\item{Using this finger, touch the surface of the water at the center of the cup or dish.}
%\end{enumerate}
%\subsubsection*{Results and Conclusions}
%When your finger touched the center of the surface of the water, the pepper moves quickly towards the edge of the water. This is because as your finger touches the surface, you introduce a small amount of soap or detergent, lowering the surface tension at that point. The surface of the water is now unbalanced; the surface tension near the edge is pulling the surface outwards more strongly than the surface tension at the center is pulling the surface inwards. As there is a net force on the surface outwards towards the edge, the surface moves, pulling the pepper along with it to the edges of the cup or dish.
%
%\subsubsection*{Clean Up Procedure}
%Dispose of the water and return all materials to their proper places.
%
%\subsubsection*{Discussion Questions}
%\begin{enumerate}
%\item{Why does the pepper move quickly away from the center?}
%\item{What happens if you add other liquids instead of soap?}
%\end{enumerate}
%
%\subsubsection*{Notes}
%Floating objects will tend towards the area with highest surface tension. The pepper in this case is following the higher surface tension towards the side of the container.


\subsection{Water Dome}
\begin{itemize}
\item{Materials: Coin, water, syringe or eyedropper}
\item{Preparation Time: none}
\item{Procedure: Place a coin flat on the table. Using the syringe or eyedropper, carefully drop individual water drops onto the coin. With some practice, you should be able to get quite a few drops onto the coin before the water spills over, creating a dome of water.}
\item{Theory: The surface tension of the water holds it together against the force of gravity, which is trying to pull the water off the coin.}
\end{itemize}


%\subsection*{Water Dome}
%
%\subsubsection*{Learning Objectives}
%\begin{itemize}
%\item{To observe the strength of surface tension and cohesion}
%\end{itemize}
%
%\subsubsection*{Background Information}
%Surface tension can be surprisingly strong. Insects and objects can float on top of water because of its tension; also water itself holds together in droplets. Combining droplets allows cohesion to form more bonds, creating a larger droplet with the same surface tension.
%
%\subsubsection*{Materials}
%Coin, water, syringe or eyedropper
%
%\subsubsection*{Activity Procedure}
%\begin{enumerate}
%\item{Place a coin on the table.}
%\item{Using the syringe or eyedropper, carefully drop individual water drops onto the coin. With some practice, you should be able to get quite a few drops onto the coin before the water spills over, creating a dome of water.}
%\end{enumerate}
%
%\subsubsection*{Results and Conclusions}
%As you add more water to the coin, the water forms a larger and larger dome rather than spilling off the coin. The surface tension of the water holds it together against the force of gravity, which is trying to pull the water off the coin. 
%
%\subsubsection*{Clean Up Procedure}
%Return all materials to their proper places.
%
%\subsubsection*{Discussion Questions}
%\begin{enumerate}
%\item{How many drops did you think would fit on the coin? How many actually did fit?}
%\item{Why does the water dome eventually break?}
%\item{Describe all the forces acting on the water.}
%\end{enumerate}
%
%\subsubsection*{Notes}
%The most dramatic results can be see on small coins, though big coins can also be used. You should be able to get at least 60 drops on a 200 shilling coin.


%Capillarity


\subsection{Capillarity}

\subsubsection*{Learning Objectives}
\begin{itemize}
\item{To observe the effect of capillarity in various liquids.} 
\item{To explain the mode of action of capillarity.} 
\end{itemize}

\subsubsection*{Background Information}
Capillarity, or capillary action, is possible because of the combined effects of cohesion and adhesion. Cohesion holds a liquid together, especially at the surface. Adhesion allows a liquid to attach itself to another material, like the vertical surface of a container. 

\subsubsection*{Materials}
Clear thin plastic straws with different diameters, shallow container (bottom of a water bottle, jar cap), various liquids like water, spirit, cooking oil, and kerosene 

\subsubsection*{Activity Procedure}
\begin{enumerate}
\item{Pour some water into the container so that it is about 1 cm deep.} 
\item{Place one end of the straw into the liquid so that the end is completely submerged but not touching the bottom of the container.} 
\item{Observe the change in water level in the straw for about a minute.} 
\item{Repeat these steps for the other liquids and compare.}
\item{Repeat these steps for straws of different diameters and compare.} 
\item{Repeat these steps for different liquids and compare.} 
\end{enumerate}

\subsubsection*{Results and Conclusions}
Liquid in a thin tube will slowly climb up the tube without any visible force present. Adhesion and cohesion are pulling the liquid up the tube. The capillary action of the liquid depends on the diameter of the tube. Different liquids have different capillary action for the same size capillary.

\subsubsection*{Clean Up Procedure}
\begin{enumerate}
\item{Return all liquids to their proper containers and put them away.} 
\item{Wash the container and straw and put them away.} 
\end{enumerate}

\subsubsection*{Discussion Questions}
\begin{enumerate}
\item{What causes the liquid to move up the straw?}
\item{What causes the liquid at the edge to cling to the straw while the liquid in the middle remains lower?}
\item{Which liquid moved up the straw fastest? Which moved slowest?}
\item{What would happen if the diameter of the straw was increased? What would happen if it was decreased?}
\end{enumerate}

\subsubsection*{Notes}
Liquids are able to climb up a thin tube due to the combined effects of adhesion and cohesion. Cohesion allows a liquid to remain connected to itself while adhesion allows a liquid to remain connected to another surface. Adhesion causes a liquid to climb slightly up the side of any container; the surface tension of the liquid (cohesion) then pulls the remainder of the liquid up as well. In a normal container, the adhesive force at the side of the container is not strong enough to pull all the other liquid up. In a thin container, a larger proportion of liquid is attached to the side of the tube and a smaller proportion is being held by surface tension, so the adhesive force is strong enough to pull all the liquid up the tube.



%Diffusion


\subsection{Lemonade}
\begin{itemize}
\item{Preparation Time: 5 minutes}
\item{Materials: lemon, drinking water, pitcher}
\item{Procedure: Make lemonade by putting lemon wedges (oranges also work) into the pitcher and adding about a liter of water. Let it sit for a couple hours, then drink and enjoy! Adding sugar or honey is recommended.}
\item{Theory: The citrus flavor of the lemons will gradually spread throughout the water, though no force is apparent. This process is called diffusion. See the Transport topic in the Biology section for more activities involving diffusion.}
\end{itemize}


\subsection{Powder Diffusion}
\begin{itemize}
\item{Preparation time: 0 minutes}
\item{Materials: powdered food coloring or kool-aid like product, water, plastic water bottle}
\item{Procedure: Fill the plastic water bottle with water. Quickly add the powdered food color, but do not shake. Observe the color diffuse through the water.}
\item{Theory: Mixing does not occur immediately. Without shaking or stirring, it occurs slowly. By using a colored compound, it is easy to see how the molecules are slowly dissolving into the solution.}
\end{itemize}

\subsection{Orange Diffusion, Part A -- Sweet Smells}
\begin{itemize}
\item{Preparation time: 5 minutes}
\item{Materials: one orange or other citrus fruit}
\item{Procedure: Have students sit in their seats. Start to peel the orange. When students begin to smell oranges, have them raise their hands. Be sure the students only raise their hands as they smell the orange and not before.}
\item{Theory: Diffusion happens in not only liquids but also gases. Peeling oranges or other citrus fruits releases small compounds that diffuse through gases. When these compounds come in contact with out noses, we smell oranges. However, we cannot smell oranges immediately on peeling; the compounds must migrate towards our noses. In this case, the compounds will slowly diffuse in the classroom with the students closest to the orange smelling it first. The students in the back of the classroom will smell it last. The effects of wind should be considered.}
\end{itemize}

\subsection{Orange Diffusion, Part B -- Trapped}
\begin{itemize}
\item{Preparation time: 5 minutes}
\item{Materials: a box, one orange or other citrus fruit}
\item{Procedure: Turn the box upside down. Without turning the box over, peel the orange inside of the box. When students begin smelling oranges, have them raise their hands.}
\item{Theory: Diffusion can only occur when the molecules can move freely. Some objects will not allow compounds through. In this activity, the cardboard box prevents the compounds in the orange to diffuse out through the classroom. This time, students not smell oranges or it will take a long time for students to start smelling.}
\end{itemize}


%Osmosis


\subsection{Osmosis}
\begin{itemize}
\item{Preparation time: 10 minutes}
\item{Materials: 1 potato or carrot, water, salt, two water bottle bottoms}
\item{Procedure: Cut two equal sized pieces of potato. Put one piece is normal water and the other in a salt-water solution. Observe over the next few hours.}
\item{Theory: In all cells and plants, there is a proper balance of different concentrations of salts and sugars. Osmosis is the process where the salts move from a high concentration either to a low concentration or where water moves from a low concentration to a high concentration. In this activity, placing the potato in pure water will cause the potato to swell. Inside the potato, there is a higher concentration of salts and sugars compared to the water surrounding it. The water moves into the potato in order to make the concentrations inside the potato more similar to the water. The potato swelling is visual evidence of this phenomenon. The potato in salt water has exactly the opposite effect. The concentration of salts inside the potato is much lower compared to the concentration of salt in the water surrounding the potato. The water in the potato moves out of the potato to dilute the salt solution.}
\end{itemize}