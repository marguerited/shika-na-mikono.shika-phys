\chapter{Effects of Pressure on the Melting Point}

\section{Aim}
To find the effect of external pressure on the melting point of ice

\section{Background Information}
Usually, the melting point of a solid is slightly increased when the external pressure is increased. This is because the increased external pressure pushes the atoms of the solid together harder. Thus, more heat must be supplied to overcome this extra squeezing and allow the atomic bonds to loosen and form a liquid. Do you think this is also true for ice?  

\section{Materials}
Block of ice, 2 masses (2 kg), 2 masses (1 kg), thin metal wire (copper\slash constantan), water trough

\section{Procedure}
\begin{enumerate}
\item Place a block of ice between two desks, over a water trough.
\item Tie 2 kg masses to each end of a thin metal wire.
\item Tie 1 kg masses to each end of another wire made of the same material as the first one.
\item Hang the wires with the weights over the block of ice apart from each other. 
\item Observe what happens to the block over the next 30 minutes.
\end{enumerate}

%\begin{figure}[h!]
%\centering
%\includegraphics[width=10cm]{./img/}
%\caption{ practical setup}
%\label{fig:}
%\end{figure}

\section{Analysis and Interpretation}
\begin{enumerate}
\item Compare the rates of melting between the areas without wires and with wires.
\item Which wire melted faster?
\item What do your observations mean about the melting point in each area?
\item According to the background information does ice act like other solid materials when under increased pressure? 
\end{enumerate}

\section{Conclusion}
From this experiment what is the effect of external pressure on the melting point of ice?

\section{Questions for Discussion}
\begin{enumerate}
\item If the same weight was hung from two wires, but one wire had a larger diameter than the other, which would melt the ice faster?
\item Give an explanation for the conclusion of this experiment.
\item Why would it be more dangerous to walk on a frozen lake of water rather than a frozen lake of oil? 
\end{enumerate}

\section{Reflection and Self Assessment}
\begin{enumerate}
\item Do you feel you have understood everything in this experiment? If not, what can you do to increase your understanding?
\item What did you find most interesting about this experiment? Explain
\item How might you use the results of this experiment in your daily life?
\end{enumerate}