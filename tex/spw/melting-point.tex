\chapter{Melting Point of a Substance}

\section{Aim}
To learn how to determine the melting point of a substance using a cooling curve

\section{Background Information}
All things in the universe can be in the three states of matter: solid, liquid and gas. Each body changes from one state to another at different pressures and temperatures. It is useful for industry, science, and daily life to know at what temperature an object will melt at atmospheric pressure. Thus, we must find some relationship between the temperature and the melting point.

\section{Materials}
Retort stand, test tube, tripod stand, wire gauze, thermometer, stop watch, beaker, naphthalene powder, water, Bunsen burner, match box

\section{Procedure}
\begin{enumerate}
\item Fill a test tube with naphthalene powder about $^3/_4$ full.
\item Support the test tube vertically using a retort stand.
\item Fill a beaker with water until it is half full.
\item Put wire gauze on top of the retort stand and place the beaker containing water on top of it.
\item Adjust the clamp of retort stand to lower the test tube with naphthalene in the beaker, ensure that the test tube does not touch the bottom of the beaker.
\item Light the Bunsen burner and heat the beaker containing water and naphthalene until the naphthalene melts.
\item Insert the thermometer into the naphthalene liquid and continue to heat until naphthalene liquid reaches a temperature of about $95^\circ$C.
\item Stop heating and remove the Bunsen burner.
\item Record the temperature at intervals of one minute as it cools from $90^\circ$C to $50^\circ$C and tabulate the results. Take note at which temperature the liquid solidifies.
\end{enumerate}

%\begin{figure}[h!]
%\centering
%\includegraphics[width=10cm]{./img/}
%\caption{ practical setup}
%\label{fig:}
%\end{figure}

\section{Safety Measure}
\begin{enumerate}
\item Naphthalene is a hydrocarbon when in liquid form so it should not be close to the fire.
\item Do not place the heated test tube containing naphthalene into cold water.
\item Do not directly inhale the naphthalene fume\slash gas because it is poisonous.
\end{enumerate}

\section{Analysis and Interpretation}
\begin{enumerate}
\item Plot a graph of temperature against time.
\item Explain the shape of the graph.
\item What is the melting point of naphthalene?
\end{enumerate}

\section{Conclusion}
Explain how to use a cooling curve to determine the melting point of a substance.

\section{Questions for Discussion}
\begin{enumerate}
\item Why was the test tube containing naphthalene heated in the water?
\item Why it is important to know the melting point of the substance before using it?
\item Why is naphthalene supposed to be heated above $80^\circ$C?
\item Water pipes in cold countries are usually wrapped in felt, what is the reason for doing this?
\end{enumerate}

\section{Reflection and Self Assessment}
\begin{enumerate}
\item How could you use the results of this experiment in you daily life?
\item What were most and least interesting about this experiment? Explain.
\end{enumerate}