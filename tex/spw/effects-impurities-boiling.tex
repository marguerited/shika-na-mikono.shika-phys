\chapter{Effects of Impurities on the Boiling Point}

\section{Aim}
To discover the effect a salt has on the boiling point of water

\section{Background Information}
Impurities are foreign substances within a pure substance. For example, pure water should have nothing in it except water molecules, but in natural water other substances such as salts, small particles of solid matter, and even dissolved gases can often be found. The effect of impurities is a very general topic and exactly how an impurity will affect the boiling point depends on the nature of the impurity and its relation to the substance being boiled. How do you think common table salt affects the boiling point of water?

\section{Materials}
Water, source of heat, beaker (250 mL), thermometer, tripod stand, wire gauze, salt, retort stand, beam balance

\section{Procedure}
\begin{enumerate}
\item Fill a beaker half full of water.
\item Heat the water in the beaker steadily until it boils, fix a thermometer on a retort stand and insert it in a beaker then record the temp of the water.
\item Add 5 g of common salt to the water. Continue heating the water and record the temperature when the temperature remains steady.
\item Continue to add salt in intervals of 5 g until 30 g total have been added. Each time record the boiling point.
\item Tabulate the results including the mass of salt in water and the temperature of the boiling point.
\end{enumerate}

%\begin{figure}[h!]
%\centering
%\includegraphics[width=10cm]{./img/}
%\caption{ practical setup}
%\label{fig:}
%\end{figure}

\section{Safety Measure}
The thermometer should not touch the bottom of the beaker.

\section{Analysis and Interpretation}
\begin{enumerate}
\item Compute the change in temperature of the boiling point from its original value.
\item Plot the graph of mass of salt added against the change in temperature of the boiling point from the original value.
\item What is the nature of the graph?
\end{enumerate}

\section{Conclusion}
From the graph what can you conclude about effects of salt on the boiling point of water?

\section{Questions for Discussion}
\begin{enumerate}
\item Why should the thermometer not touch the bottom of the beaker?
\item Do you expect a change in the boiling point temperature if you added sand instead of salt? Explain.
\item What change in boiling temperature do you expect if you mixed in ethanol, which has a lower boiling point than water?
\end{enumerate}

\section{Reflection and Self Assessment}
\begin{enumerate}
\item Did you face any problems during this experiment? If yes, how could you solve them?
\item How can you use the results of this experiment to cook food faster?
\end{enumerate}