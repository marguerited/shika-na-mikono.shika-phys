\section{Archimedes' Principle and the Law of Flotation}


%Eureka Can


\subsection{Construction of Eureka Can}

\subsubsection*{Learning Objectives}
\begin{itemize}
\item{To construct a eureka can and use it to measure volumes of irregular objects.} 
\end{itemize}

\subsubsection*{Materials}
Plastic bottles of 500 mL or 1000 mL, straws* or a syringe needle cap, a knife/razor blade, heat source*, cellotape or super glue, nail, and a wire

\subsubsection*{Preparation Procedure}
\begin{enumerate}
\item{Cut the plastic bottle in half and use the bottom part.} 
\item{Heat a sharp pointed end of the nail.} 
\item{Using the heated sharp point of the nail, make a small hole about 5 cm from the top of the cut plastic bottle.} 
\item{Cut a piece of straw about 5 cm long or use the syringe needle cap.} 
\item{Insert the piece of straw into the hole. Make sure the piece of straw is tightly fixed with cellotape or super glue.} 
\end{enumerate}

\subsubsection*{Activity Procedure}
Use the constructed eureka can to overflow different liquids and to measure the volume of liquid displaced when an object floats in it.

\subsubsection*{Clean Up Procedure}
Collect all the used materials, cleaning and storing items that will be used later.

\subsubsection*{Discussion Questions}
\begin{enumerate}
\item{What is the relationship between the weight of a floating object and the weight of water it displaces?}
\item{What is the reason for using a Eureka Can instead of a normal bottle?}
\end{enumerate}

\subsubsection*{Notes}
A Eureka Can is typically used to move a certain volume from the can into another container.  When an object floats in a liquid, it displaces its own weight in water.  If the level of water in the Eureka can is at the level of the spout, the water displaced will flow through the spout into another container.  This water can then be measured on a beam balance to find the weight of the object.



\subsection{Water Weight and Upthrust}
\begin{itemize}
\item{Preparation Time: 1 minute}
\item{Materials: spring balance, syringe with the bottom melted shut and no plunger, eureka can (can be made cheaply by a metal craftsman), water, heavy object, thread, small dish}
\item{Procedure: Fill the eureka can up to its spout with water and place the spout over the dish. This can is designed so that when the water is being displaced, it is collected to another container for later measurements. Hang the object by the thread from the spring balance and measure its weight. Now immerse the water completely in the eureka can and measure its Apparent Weight (weight in water).\\
When you immersed the object in water, some water will have overflowed from the can into the small dish. Pour this water into your syringe shell and measure the weight of water. Record this result with the earlier Weight and Apparent Weight.}
\item{Theory: Archimedes’ Principle states that the upthrust of a liquid on an object is equal to the weight of water displaced by the object. The upthrust is equal to the Weight of the object minus its Apparent Weight in the water:@	Upthrust = Weight in air – Apparent Weight in liquid\\
But upthrust is also equal to the water displaced:@	Upthrust = Weight of liquid displaced\\
By calculating the upthrust, you should see that the result is equal to the weight of water in the syringe.}
\end{itemize}


%Sinking and Floatinng


%\subsection{Floating Eggs}
%\begin{itemize}
%\item{Preparation time: 5 minutes}
%\item{Materials: 1 uncooked egg, 1 jam jar, water, salt}
%\item{Procedure: Fill jam jar with water. Place egg in the water. The egg will sink. Add salt until the egg floats.}
%\item{Theory: The density of an egg is greater than water. This is why the egg will sink. Since density is defined as mass divided by volume, the density of water can be changed by dissolving extra mass, salt in this case. As more and more salt dissolves in the water, the densities increases until the density of the water is greater than the egg and the egg floats to the surface.\\
%This is the same reason why it is more difficult to swim in fresh water than salt or ocean water. The extra salt or ions in the ocean water increase its density and making the body more buoyant. Since those ions are much fewer in fresh water, the density of fresh water is greater than salt water. The Dead Sea in the Middle East has so much salt in the water that people do not swim in the water; they just float. Lake Natron in Tanzania probably has similar properties.}
%\end{itemize}



\subsection{Conditions of Flotation}

\subsubsection*{Learning Objectives}
\begin{itemize}
\item{To explain the conditions for a floating body.} 
\end{itemize}

\subsubsection*{Materials}
Water, fresh egg, salt, beaker*

\subsubsection*{Activity Procedure}
\begin{enumerate}
\item{Fill a beaker with water.} 
\item{Release a fresh egg on the surface of water slowly and observe its position.} 
\item{Add some salt while stirring and observe the position of the egg.} 
\item{Continue adding salt and observe the position of the egg.} 
\end{enumerate}

\subsubsection*{Results and Conclusion}
The egg sinks to the bottom of the container because its density is greater than that of water. After adding the salt, the egg rises and finally floats on the surface of water. This is because the density of the water becomes higher than that of the egg.  

\subsubsection*{Clean Up Procedure}
Collect all the used materials, cleaning and storing items that will be used later.

\subsubsection*{Discussion Questions}
\begin{enumerate}
\item{Why does the egg sink initially?}
\item{What causes the egg to float after the addition of salt?}
\end{enumerate}

\subsubsection*{Notes}
An object immersed in water experiences upward force equal to the weight of the water it displaces. The upthrust competes with the downward pull of gravity which diminishes the weight of the object. If the upthrust is greater than the object's weight, it will float. Otherwise, the object will sink. If the water density is greater than the average density of the object, the object will also float. If the water density is less than the average density of the object, the object will sink.  


%Law of Flotation


%Hydrometer


\subsection{Construction and Use of a Hydrometer}

\subsubsection*{Learning Objectives}
\begin{itemize}
\item{To construct a simple hydrometer.} 
\item{To explain the use of a hydrometer.} 
\item{To calibrate a hydrometer.} 
\item{To use a hydrometer to measure the density of various liquids.} 
\end{itemize}

 \begin{figure}
\begin{center}
\def\svgwidth{150pt}
\input{./img/hydrometer.pdf_tex}
\caption{A Hydrometer}
\label{fig:hydrometer}
\end{center}
\end{figure}

\subsubsection*{Background Information}
Each liquid has a different density. The level at which an object will float in a liquid depends on the density of that liquid, so the different densities of liquids can be observed and measured by observing the level at which an object floats in them.  

\subsubsection*{Materials}
Plastic straw, small piece of plastic bag, dry sand, several containers to hold liquids, marker pen, rubber band or thread, ruler, water, kerosene, honey, any other liquids

\subsubsection*{Preparation Procedure}
\begin{enumerate}
\item{Close one end of the straw with the plastic bag and secure it with the rubber band or thread so that water cannot enter.} 
\item{Place the straw in water with the plastic bag side down.} 
\item{Pour sand into the straw until the bottom is heavy enough that the straw floats upright.} 
\end{enumerate}

\subsubsection*{Activity Procedure}
\begin{enumerate}
\item{Place the straw in water so that it floats upright without touching the bottom or leaning.} 
\item{Use the marker pen to mark the water level on the outside of the straw. Label this mark as 1.0 (the known density of water).} 
\item{Place the straw in a container of kerosene so that it floats upright without touching the bottom or leaning to one side.} 
\item{Use the marker to mark the kerosene level on the outside of the straw. Label this mark as 0.8 (the known density of kerosene).}
\item{Remove the straw from the kerosene and clean it. Be careful not to get any liquid inside the straw.} 
\item{Use a ruler to draw an accurate scale on the straw, using the 1.0 and 0.8 marks as starting points. Begin by making a mark directly between them as 0.9, etc.} 
\item{When the scale is complete (both above 1.0 and below 0.8), use the designed hydrometer to measure the densities of other liquids by reading the mark at the level of the liquid.} 
\end{enumerate}

\subsubsection*{Results and Conclusions}
The straw, when properly weighted, will float upright in any liquid and will therefore provide a good surface to write levels of liquids. The density of water is known as 1.0 and is relatively constant. Kerosene is also known as 0.8 and will not vary from place to place. By writing both of these points on the hydrometer and the respective floating levels, we can create a scale extending up from 1.0 and down from 0.8. This can then be used to measure the densities of other liquids.  

\subsubsection*{Clean Up Procedure}
\begin{enumerate}
\item{Return all liquids to their proper containers.} 
\item{Clean the outside of the hydrometer.} 
\item{Return all materials to their proper places.} 
\end{enumerate}

\subsubsection*{Discussion Questions}
\begin{enumerate}
\item{What liquids have high density?}
\item{What liquids have low density?}
\item{Why, when reading a hydrometer, do the small numbers appear at the top of the scale?}
\end{enumerate}

\subsubsection*{Notes}
Be careful not to get any liquid in the straw. If the sand becomes wet, the hydrometer will not work again until it dries.  