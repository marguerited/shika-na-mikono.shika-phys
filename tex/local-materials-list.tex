\chapter{Local Materials List}
\label{cha:local-materials}

In order to gain a thorough understanding of science, students must be able to make a connection between classroom learning and the outside world. The following is a list of locally available materials which may be used to substitute conventional materials and apparatus for various activities. These materials have the following advantages: 
\begin{itemize*}
\item They are readily available in the village or a nearby town;
\item They are cheaper than conventional materials; 
\item They may safely substitute the conventional materials without fear of losing accuracy or understanding; 
\item They help students to draw a connection between science education and the world around them.
\end{itemize*}
Imagination and innovativeness is encouraged on the part of the student and teacher to find other suitable local substitutions. \\

\noindent Throughout this book you will see materials that have been marked with an asterisk (*). These are items which may be made or purchased using locally available substitutes. The guide for using and making these local materials is found in this section.  

\section*{Alligator Clips}
\vspace{-10pt}
\textbf{Use:} Connecting electrical components\\
\textbf{Materials:} Clothespins, aluminum foil, glue\\
\textbf{Procedure:} Glue aluminum foil around the clamping tips of a clothespin.

\section*{Balance}
\vspace{-10pt}
See the Form I activity on \nameref{sub:beambalance}.

\section*{Beakers}
\vspace{-10pt}
\textbf{Use:} To hold liquids, to heat liquids\\
\textbf{Materials:} Water bottles, jam jars, metal cans, knife\slash razor\\
\textbf{Procedure:} Take empty plastic bottles of different sizes. Cut them in half. The base can be used as a beaker. Jam jars made of glass or cut off metal cans may be used when heating. 

\section*{Bunsen Burner}
\vspace{-10pt}
See \nameref{sec:heatsources}.

\section*{Circuit Components}
\vspace{-10pt}
\textbf{Use:} Building simple circuits, Ohm's Law, amplifier, wave rectifiers\\
\textbf{Materials:} Broken radio, computer, stereo, other electrical devices\\
\textbf{Procedure:} Remove resistors, capacitors, transistors, diodes, motors, wires, transformers, inductors, rheostats, pulleys, gears, battery holders, switches, speakers and other components from the devices. Capacitors tend to state their capacitance in microFarads on their bodies.

\section*{Delivery Tube}
\label{sec:delivery-tube}
\vspace{-10pt}
\textbf{Use:} For the movement and collection of gases, capillary tubes, hydraulic press\\
\textbf{Materials:} Straws, pen tubes, IV tubing (giving sets) from a pharmacy, bicycle tubing, or pawpaw petioles

\section*{Drawing Board}
\vspace{-10pt}
\textbf{Use:} Reflection, refraction of light\\
\textbf{Materials:} Thick cardboard

\section*{Droppers}
\vspace{-10pt}
\textbf{Use:} To transfer small amounts of liquid \\
\textbf{Materials:} 2 mL syringes\\
\textbf{Procedure:} Take a syringe. Remove the needle to use as a dropper. 

\section*{Eureka Can}
\label{sec:eureka-can}
\vspace{-10pt}
\textbf{Use:} To measure volume of an irregular object, Archimedes' Principle, Law of Flotation\\
\textbf{Materials:} Plastic bottle, knife, Optional: super glue, straw, nail, candle\\
\textbf{Procedure:} Cut the top off of a 500 mL plastic bottle. Then cut a small strip at the top (1 cm wide by 3 cm long) and fold down to make a spout. Alternatively, heat a nail using a candle and poke a hole near the top of a cut off bottle. Super glue a straw so that it fits securely in the hole without leaking.

\section*{Funnel}
\vspace{-10pt}
\textbf{Use:} To guide liquid or powder into a small opening\\
\textbf{Materials:} Empty water bottles, knife\\
\textbf{Procedure:} Take an empty water bottle and remove the cap. Cut it in half. The upper part of the bottle can be used as a funnel.  

\section*{Glass Blocks}
\vspace{-10pt}
\textbf{Use:} Refraction of light\\
\textbf{Materials:} 8~mm - 15~mm slabs of glass\\
\textbf{Procedure:} Have a craftsman make rectangular pieces of glass with beveled edges, so students do not cut themselves. Glass blocks from a lab supply company are generally 15~mm thick. 8~mm and 10~mm glass is relatively common in towns. 12~mm and thicker glass exists though is even more difficult to find. 
Stack several pieces of thinner glass together and turn them on their edge.

\section*{Heat Source}
\label{sec:heatsources}
\vspace{-10pt}
\textbf{Use:} Heating substances\\
\textbf{Materials:} Candles, kerosene stoves, charcoal burners, Motopoa (alcohol infused heavy oil), metal can, bottle caps, butane lighter \\
\textbf{Procedure:} Cut a metal can in half or use a bottle cap and add a small amount of Motopoa.

\section*{Iron Filings}
\vspace{-10pt}
\textbf{Use:} To map magnetic fields\\
\textbf{Materials:} Steel wool / Iron wool used for cleaning pots\\
\textbf{Procedure:} Rub some steel wool between your thumb and fingers.  The small pieces that fall are iron filings.  Collect them in a matchbox or other container to use again.

\section*{Light Bulbs}
\vspace{-10pt}
\textbf{Use:} Electrical circuits, diodes\\
\textbf{Materials:} Broken phone chargers, flashlights, other electronic devices\\
\textbf{Procedure:} Look for LEDs from broken items at hardware stores, local technicians, or small shops. 

\section*{Masses}
\label{sec:masses}
\vspace{-10pt}
\textbf{Use:} Calibrating and using beam balance and spring balance, Hooke's Law\\
\textbf{Materials:} Known masses, beam balance, empty bottles, plastic syringe, water, plastic bags, sand, stones, thread, paper, tape, pen\\
\textbf{Procedure:} Use a beam balance and known masses at a market or nearby school to measure exact masses of sand or stones.  Use a marker pen to mark the masses on the stones. 

If using sand, place a small piece of plastic bag on the scale pan and fill it with sand until you have the required mass.  Tie the sand in the plastic bag with thread.  Use paper and tape to make a label on the outside, marking the mass with pen.  These masses can be used in your school.

If using water, use a beam balance from a nearby school to measure the exact mass of an empty water bottle. Add a volume of water in mL equal to the mass in g needed to reach a desired total mass. (The density of water is 1.0 g/mL, so you can use a known volume of water in a bottle to create a known mass.) This can be done precisely by using a plastic syringe. Label the bottle with tape and a pen.

\section*{Measuring Cylinder}
\label{sec:meascyl}
See the Form I activity on \nameref{sub:meascyl}.

\section*{Metre Rule}
\vspace{-10pt}
\textbf{Use:} Measuring length, Principle of Moments, drawing graphs\\
\textbf{Materials:} Slabs of wood, ceiling board, permanent pen\\
\textbf{Procedure:} Buy one, take it and a permanent pen to a carpenter, and leave with twenty. Measure each new one to the original rule to prevent compounding errors. See also the Form I activity on \nameref{sub:metrerule}.

\section*{Nichrome Wire / Resistance Wire}
\vspace{-10pt}
See \nameref{sec:wire}.

\section*{Optical Pins}
\vspace{-10pt}
\textbf{Use:} Compass needles, making holes, flying wire\\
\textbf{Materials:} Office pins, sewing needles, needles from syringes

\section*{Plane Mirror}
\vspace{-10pt}
\textbf{Use:} Laws of Reflection, periscope, water prism, super glue, small wooden blocks\\
\textbf{Materials:} piece of thin glass, kibatari, Optional: small pieces of mirror glass are cheap or free at a glass cutter's shop\\
\textbf{Procedure:} Light the kibatari so that it creates a lot of smoke.  Pass one side of the glass repeatedly over the kibatari until that side is totally black.  The other side acts as a mirror. Super glue to small wooden blocks to stand upright.

\section*{Resistors}
\label{sec:resistors}
\vspace{-10pt}
\textbf{Use:} Electrical components\\
\textbf{Materials:} Old radios, circuit boards, soldering iron\\
\textbf{Procedure:} Remove resistors from old radios and circuit boards by melting the solder with a soldering iron or a stiff wire heated by a charcoal stove. If you need to know the ohms, the resistors tell you. Each has four strips (five if there is a quality band) and should be read with the silver or gold strip for tolerance on the right. Each color corresponds to a number:\\

\begin{tabular}{lll}
black = 0 & yellow = 4 & violet = 7\\
brown = 1 & green = 5 & gray = 8\\
red = 2 & blue = 6 & white = 9\\
orange = 3 & & \\[10pt]
\end{tabular} 

and additionally for the third stripe: gold = -1 and silver = -2. \\

\noindent The first two numbers should be taken as a two digit number, so green-violet would be 57, red-black 20, etc. The third number should be taken as the power of ten (a $ 10^{n} $ term), so red-orange-yellow would be $ 23 \times 10^{4} = 230000 $, red-brown-black would be $ 21 \times 10^{0} = 21 $ and blue-gray-silver would be $ 68 \times 10^{-2} = 0.68 $. The unit is always ohms. The fourth and possibly fifth bands may be ignored.

\section*{Retort Stand}
\label{sec:retort-stand}
\vspace{-10pt}
\textbf{Use:} To hold pendulums, to elevate springs or other objects\\
\textbf{Materials:} Filled 1.5 L water bottle, straight bamboo stick, tape, marker\\
\textbf{Procedure:} Tape the bamboo stick across the top of the water bottle so that it reaches out ~20 cm to one side. Attach a small clamp if required or hang object directly from bamboo stick.

\section*{Scale Pan}
\label{sec:scale-pan}
\vspace{-10pt}
\textbf{Use:} Beam balance, Hooke's Law\\
\textbf{Materials:} Plastic bottle, cardboard box, string\\
\textbf{Procedure:} Cut off the bottom of a plastic bottle or cardboard box. Poke 3 or more holes near the top and tie string through each hole. Join strings and tie at the top to hang from a single point.

\section*{Spring Balance}
\label{sec:spring-balance}
\textbf{Use:} To measure force applied on an object\\
\textbf{Materials:} Strip of cardboard, rubber band, 2 paper clips, staple pin, pen\\
\textbf{Procedure:} Cut a rubber band and fix one end to the top of a cardboard strip using a staple pin. (A stronger rubber band allows for a greater range of forces to measure.) Attach one paper clip near the top as a pointer. Attach the other paper clip as a hook at the bottom of the rubber band. Calibrate the spring balance using known masses. Write the equivalent force in Newtons on the cardboard. (A 1 g mass has a weight of 0.01 N, 100 g has a weight of 1 N, etc.)

\section*{Springs}
\vspace{-10pt}
\textbf{Use:} Hooke's Law, potential energy, work, spring balance\\
\textbf{Materials:} Springs from hardware stores, bike stores, junk merchants in markets, window blinds, rubber bands, strips of elastic\\
\textbf{Procedure:} Remove plastic covering if necessary and cut to a desired length (~5 cm). Alternatively use rubber bands or elastic from a local tailor - these can also be used to calculate a constant of elasticity.

\section*{Stopper}
\vspace{-10pt}
\textbf{Use:} To cover the mouth of a bottle, hold a capillary tube\\
\textbf{Materials:} Rubber from old tires or sandals, cork, plastic bottle cap, pen tube, super glue\\
\textbf{Procedure:} Cut a circular piece of rubber.  If the stopper is being used to hold a capillary tube, a hole can be melted in a plastic cap or rubber stopper. Alternatively, super glue a pen tube to a plastic bottle cap and connect to rubber tubing.

\section*{Stopwatches}
\vspace{-10pt}
\textbf{Use:} Simple pendulum, velocity, acceleration\\
\textbf{Materials:} Athletic and laboratory stopwatches from markets, digital wristwatches

\section*{Test Tubes}
\label{sec:testtubes}
\vspace{-10pt}
\textbf{Use:} To heat materials without a direct flame, to combine solutions\\
\textbf{Materials:} 10~mL syringes, matches\\
\textbf{Procedure:} Remove the needle and plunger from 10~mL syringes. Heat the end of the shell with a match until it melts. Press the molten end against a flat surface (like the end of the plunger) to fuse it closed. If the tube leaks, fuse it again. Test tubes made this way may be heated in a water bath up to boiling, hot enough for most experiments.

\section*{Test Tube Holder / Tongs}
\vspace{-10pt}
\textbf{Use:} To handle test tubes\\
\textbf{Materials:} Wooden clothespins, stiff wire, strip of paper or cloth\\
\textbf{Procedure:} Use clothespins or stiff wire for prolonged heating, or strips of paper or cloth for short-term heating.

\section*{Test Tube Racks}
\vspace{-10pt}
\textbf{Use:} To hold test tubes vertically in place\\
\textbf{Materials:} Wire grid from local gardening store, styrofoam block, plastic bottle, knife\\
\textbf{Procedure:} Fold a sheet of wire grid to make a table; punch holes in a piece of styrofoam; cut a plastic bottle in half and fill it with sand to increase stability. Or cut a plastic bottle along its vertical axis and rest the two cut edges on a flat surface. Cut holes into it for the test tubes.

\section*{Tripod Stands}
\vspace{-10pt}
\textbf{Use:} For supporting containers above heat sources, for elevating items\\
\textbf{Materials:} Stiff wire, metal rods\\
\textbf{Procedure:} Bring a sample to a welder or metal worker in town; make sure the stand is not too short or too tall. You can also make your own from stiff wire.

\section*{Water Bath}
\label{sec:hotwaterbathes}
\vspace{-10pt}
\textbf{Use:} To heat substances without using a direct flame\\
\textbf{Materials:} \nameref{sec:heatsources}, water, cooking pot\\
\textbf{Procedure:} Bring water to a boil in a small aluminum pot, then place the test tubes in the water to heat the substance inside the test tube. Prevent test tubes from falling over by clamping with clothespins or placing parallel wires across the container.

\section*{Wire}
\label{sec:wire}
\subsection*{All-purpose wire}
\vspace{-6pt}
\textbf{Use:} Connecting circuit components, current electricity\\
\textbf{Materials:} Speaker wire, knife\\
\textbf{Procedure:} Speaker wire can be found at any hardware store or taken from old appliances - the pairs of colored wires brained together. Strip using a knife, your teeth, or a wire stripper.

\subsection*{Specific gauge wire}
\vspace{-6pt}
\textbf{Use:} Electrical components, motors, transformers, simple generators\\
\textbf{Materials:} Copper wire without plastic covering (transformer wire), knife\slash scissors, matches\\
\textbf{Procedure:} Scrape or burn off the insulating varnish at any points you wish to make electrical contact. 
These wires come in a variety of diameters (gauges). A useful chart for converting diameter to gauge may be found \href{http://www.dave-cushman.net/elect/wiregauge.html}{here}. If the wire is sold by weight, 
you can find the length if you know the diameter - the density of copper metal at room temperature is 8.94~g/cm$^{3}$. For example, with 0.375~mm wire, 250~g is about 63 meters.

