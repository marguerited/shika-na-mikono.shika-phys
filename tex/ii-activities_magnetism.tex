\section{Magnetism}

%Concept of Magnetism

	%Magnetic and Non-Magnetic Materials
	
\subsection{Magnetic and Non-magnetic Materials}

\subsubsection*{Learning Objectives}
\begin{itemize}
\item{To identify magnetic and non-magnetic materials} 
\item{To observe the effect of a magnet on magnetic and non-magnetic materials} 
\end{itemize}

\subsubsection*{Background Information}
The only naturally occurring magnet is Lodestone, which is a kind of mineral. However, we tend to use iron to make magnets as it is easily magnetized and very strong. Magnets can have a very strong attractive or repulsive force, but the force only acts on some materials.

\subsubsection*{Materials}
Magnet*, various objects in the classroom or home like iron nails, plastic, wood, cloth, metal, etc.  


\subsubsection*{Activity Procedure}
\begin{enumerate}
\item{Bring the magnet close to an object.} 
\item{Observe if the object moves or if it is difficult to remove the magnet.} 
\item{Repeat this for any object you or the students can find in the classroom.} 
\item{Keep a list of what is attracted by the magnet and what is not.} 
\end{enumerate}

\subsubsection*{Results and Conclusions}
Many metals are attracted to magnets. Most other materials are not attracted to magnets.


\subsubsection*{Discussion Questions}
\begin{enumerate}
\item{What objects were attracted to the magnet?}
\item{In general, what materials are attracted to magnets?}
\item{What objects were not attracted to the magnet?}
\item{In general, what materials are not attracted to magnets?}
\end{enumerate}

\subsubsection*{Notes}
Note that some common metals (e.g. copper and aluminium) are not attracted to magnets. Also note that some rare non-metals (e.g. liquid oxygen) are attracted to magnets. Of the elements students learn about in ordinary level, iron makes the strongest magnets.
	
	%Properties of Magnets
	
\subsection{Suspended Magnet Compass}
\begin{itemize}
\item{Preparation Time: 1 minute}
\item{Materials: thread, bar magnet, Optional: second magnet}
\item{Procedure: Tie the thread around the bar magnet’s center so that it hangs horizontally and is free to spin. Allow it to settle and you will see that it points north and south. Turn it away and allow it to settle again. Rotate your hand and the magnet will stay facing north and south. If you have a second magnet, pass it by the suspended magnet and watch the suspended magnet try to face the other magnet. Take the magnet away and the suspended magnet will return to its original direction.}
\item{Theory: A magnet will naturally align itself with the Earth’s magnetic field. Usually there is too much friction for this to happen, but a suspended magnet is free to face North and South. Even if you try to confuse it by turning it or by bringing another magnet close, it will eventually align itself with the earth’s field.}
\end{itemize}
	

%Magnetisation, Demagnetisation

\subsection{Magnetizing a Nail}
\begin{itemize}
\item{Preparation Time: 1 minute}
\item{Materials: nail, insulated wire (speaker wire), 2 or more D-cell batteries}
\item{Procedure: coil the middle of the wire around the nail to create a solenoid. Connect the two ends of the wire to the battery. The nail and connectors will become hot and the nail will become magnetized. You can use it to pick up staples, paper clips, etc.}
\item{Theory: The moving electric charge in the wire solenoid creates a magnetic field in the nail (use the RHR), aligning the “domains” in north-south. The stronger the current is, the stronger the magnetic field and therefore the stronger the magnet. If you use another material, you will find that the magnet is not as strong as the iron nail.}
\end{itemize}

%Magnetic Fields

%\subsection{Mapping Magnetic Fields}
%\begin{itemize}
%\item{Preparation Time: 1 minute}
%\item{Materials: bar or horseshoe magnet, iron wool, piece of white paper}
%\item{Procedure: Place the magnet on a table and the paper over the magnet. Using your thumb and forefinger, rub the iron wool above the paper. Small pieces of iron should fall onto the paper, gradually mapping out the field of the magnet below. Move the wool around as you do this to try to show the field in a wide area. If the magnet is too strong, put some space between it and the paper. Try this with two magnets, showing attraction and repulsion between the poles. Note that the field is strongest at the poles.}
%\item{Variation: Pour the filings into a container of viscous fluid (play around with glycerin and others). Shake the container so the filings are distributed around the fluid. Hold a magnet next to the container; the filings will arrange themselves into the 3D pattern of the field.}
%\item{Theory: Magnetic fields extend from the North Pole of a magnet to any South Pole. Iron filings are small enough that they can form patterns in any magnetic field, showing the shape and the relative strength and various points on the field. If the field is strong enough, the filings will also form a 3D structure.}
%\end{itemize}


\subsection{Magnetic Fields}

\subsubsection*{Learning Objectives}
\begin{itemize}
\item{To map magnetic field lines of a given magnet using iron filings}
\item{To explain the shape and direction of magnetic fields on a magnet or between magnets}
\end{itemize}

\subsubsection*{Background Information}
Magnetic fields are invisible lines of force that run between the poles of a magnet (from North to South) or between the poles of multiple magnets.  We cannot see these lines, but we can feel them when we bring magnets close to each other.

\subsubsection*{Materials}
Steel wool, magnets*

\subsubsection*{Preparation Procedure}
\begin{enumerate}
\item{Place a sheet of paper on the table}
\item{Take the steel wool and rub it over the sheet of paper.  Small pieces (filings) of the steel wool will fall to the paper.}
\end{enumerate}

\subsubsection*{Activity Procedure}
\begin{enumerate}
\item{Place a magnet on the table.}
\item{Place a piece of paper over the magnet so that the paper is flat}
\item{Slowly and gently drop the iron filings on the paper.  Spread them evenly.}
\item{Observe the positions of the filings and the shape they create.}
\item{If another magnet is present, place it near the other magnet under the paper.}
\item{Drop more filings on the paper to observe the shape of the field between the two magnets.}
\item{Repeat this process for the two magnets in various positions: repelling, attracting, etc.}
\end{enumerate}

\begin{figure}
\begin{center}
\def\svgwidth{150pt}
\input{./img/magnetic-fields.pdf_tex}
\caption{Observing Magnetic Field Lines}
\label{fig:magnetic-fields}
\end{center}
\end{figure}

\subsubsection*{Results and Conclusions}
The iron filings are magnetic and light-weight, so they will move to follow the magnetic lines of force on a magnet.  The filings will clearly show the curved lines around the magnet from one pole to the other.  If the magnet is strong enough, the filings will show the three-dimensional field.
Between magnets, the filings will show a strong concentration of force between opposite poles and a neutral point between like poles.

\subsubsection*{Clean Up Procedure}
\begin{enumerate}
\item{Store iron filings inside a \textit{dry} bottle after using them}
\item{Return the magnets to a safe storage place.}
\end{enumerate}

\subsubsection*{Discussion Questions}
\begin{enumerate}
\item{Describe the shape of the magnetic field lines}
\item{Where is the magnetic force strongest?  Where is it weakest?}
\end{enumerate}

\subsubsection*{Notes}
You can visit any garage in town where there is a welder or metal saw and collect iron dust and use them as iron filings.


%Earth's Magnetic Field

%\subsection{Pin compass}
%\begin{itemize}
%\item{Preparation Time: 1 minute}
%\item{Materials: small pin, magnet, small dish of water}
%\item{Procedure: Magnetize the pin by stroking it with one pole of the magnet; use this time also to review methods of magnetization. Place the pin gently on the surface of the water so that it does not sink (you can review surface tension here if you like), watch as it rotates to face north and south.}
%\item{Theory: The earth is a magnet and its field lines can be seen using a compass, as a compass itself is a magnet and will align itself with any magnetic field. By magnetizing the pin, you make it into a compass needle, which will naturally align with the earth’s field, and the water allows it to pivot freely.}
%\end{itemize}


\subsection{Creating a Simple Compass}

\subsubsection*{Learning Objectives}
\begin{itemize}
\item{To construct a compass and understand its mode of action}
\item{To observe the presence of Earth's magnetic field} 
\end{itemize}

\subsubsection*{Background Information}
The earth is a magnet.  Its magnetic field lines extend from its North pole (near the geographic South pole) to its South pole (near the geographic North pole).  However, we cannot see these lines as they are simply lines of force.
Any magnet feels the force of the earth's magnetic field and tries to turn to face North.  If a light-weight magnet is allowed to rotate freely, it will turn, thus showing the direction of North and South.

\subsubsection*{Materials}
Needle or pin, magnet, small plastic lid, water, small piece of paper, and a small plastic lid.

\subsubsection*{Preparation Procedure}
\begin{enumerate}
\item{Collect the needle or pin.  If it has a heavy end, break it so that it is uniform.} 
\item{If needed get the bar magnet from a broken radio or speaker from the radio repair shop.}
\end{enumerate}

\subsubsection*{Activity Procedure}
\begin{enumerate}
\item{Rub one side of the needle or pin on a bar magnet in one direction several times, do not scratch it.} 
\item{Pour some water into the can cap.} 
\item{Stitch the magnetized pin or needle into the piece of paper and place them on the surface of the water, let it rest. Observe it.} 
\item{Give a slight push to the piece of paper so that it rotates slowly. Observe it.} 
\end{enumerate}

\subsubsection*{Results and Conclusions}
A magnetized pin or needle always comes to rest in the North-South direction.  This implies that the needle is pointing in the direction of earth's magnetic field towards the geographical north pole.

\subsubsection*{Clean Up Procedure}
Return all materials to their proper places and put the magnet in a safe place.

\subsubsection*{Discussion Question}
Which direction does the magnetized pin always point? Why?

\subsubsection*{Notes}
When magnetizing the pin or needle make sure you rub it only in one direction. Do not rub back and forth.


%\subsection{Magnetic Dip Gauge}
%\begin{itemize}
%\item{Preparation Time: 15 minutes}
%\item{Materials: magnet, sewing needle, cork, two pins, paper, pen, cardboard or metal strip}
%\item{Procedure: Push the two pins into the ends of the cork to create an axle. Push the sewing needle through the cork perpendicular to the axle pins so that the needle rolls end-over-end when you roll the cork/pins between your fingers. Adjust the needle so that it rests horizontally when the cork is free to pivot (equilibrium). Use the magnet to magnetize the needle without changing its position in the cork.\\
%Bend the metal or cardboard strip into a U-shape to create a stand for the cork and pins. Rest the pins on each style of the U-stand so that the needle is free to rotate vertically. If you like, cut out a semicircular piece of paper and label the angles 0 – 90 degrees on it; tape or glue this to the stand. The needle will rotate, or dip, to point in the vertical direction of the earth’s magnetic field.}
%\item{Theory: Earth’s magnetic field is not level across the surface of the earth: it goes into or out of the ground at an angle depending on the latitude. The angle of the field relative to the surface is called Magnetic Dip and is measured with this needle.}
%\end{itemize}


\subsection{Magnetic Dip Gauge}

\subsubsection*{Learning Objectives}
\begin{itemize}
\item{To observe the presence of magnetic dip}
\item{To measure magnetic dip}
\end{itemize}

\subsubsection*{Background}
The earth's magnetic field extends from near the geographic south pole to the near geographic north pole.  However, its lines of force pass through the surface of the earth because the lines are not perfect circles around the earth.  Where the field passes through the surface of the earth, it has a certain angle which we call the magnetic dip, or the angle of the field lines relative to the earth's surface.

\subsubsection*{Materials}
magnet, sewing needle, cork or foam, two pins, paper, pen, cardboard or metal strip

\subsubsection*{Preparation Procedure}
\begin{enumerate}
\item{Push the two pins into the ends of the cork to create an axle.}
\item{Push the sewing needle through the cork perpendicular to the axle pins so that the needle rolls end-over-end when you roll the cork/pins between your fingers.}
\item{Adjust the needle so that it rests horizontally when the cork is free to pivot (equilibrium).}
\item{Use the magnet to magnetize the needle without changing its position in the cork.  You can do this by stroking the needle in one direction on the magnet.}
\item{Bend the metal or cardboard strip into a U-shape to create a stand for the cork and pins.}
\item{Rest the pins on each vertical side of the U-stand so that the needle is free to rotate vertically.}
\item{Cut out a semicircular piece of paper and label the angles 0 to 90 degrees on it; tape or glue this to the stand.}
\end{enumerate}

\begin{figure}
\begin{center}
\def\svgwidth{150pt}
\input{./img/magnetic-dip-gauge.pdf_tex}
\caption{Magnetic Dip Gauge}
\label{fig:magnetic-dip-gauge}
\end{center}
\end{figure}

\subsubsection*{Activity Procedure}
\begin{enumerate}
\item{Set up the magnetic dip gauge so that the needle is free to rotate vertically.}
\item{Measure the angle of the needle relative to the ground.}
\end{enumerate}

\subsubsection*{Results and Conclusions}
Before magnetizing the needle, it will be able to balance horizontally in equilibrium.  However, when the needle is magnetized, it will dip down to show the direction of the earth's field.  Like a compass, the needle naturally moves to show the direction of the earth's magnetic field.

\subsubsection*{Cleanup Procedure}
Return all materials to their proper places and put the magnet in a safe place.

\subsubsection*{Discussion Questions}
\begin{enumerate}
\item{What is the direction of the needle?}
\item{Why does the needle not point horizontally, as it did before it was magnetized?}
\item{What is the angle between the needle and the ground?}
\end{enumerate}

\subsubsection*{Notes}
The magnetic dip gauge works only when it is facing North and South.  If it is facing East/West, the magnetic field will be moving perpendicular to the poles of the gauge, so it will not be able to show the correct direction.  Also, you may need to turn the gauge around if it is not showing the dip; if the needle is magnetized opposite to the direction of the earth's field, it will fail to show the correct direction.
