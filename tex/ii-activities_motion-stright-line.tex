\section{Motion in a Straight Line}

\subsection{Object Toss}
\begin{itemize}
\item{Preparation time: none}
\item{Materials: Any object(s)}
\item{Procedure: When teaching projectile motion, it is productive to throw objects in the classroom. This is useful, and extremely simple. Almost any object may be used. In the past, I have used the keys from my pocket, lemons from the lemon tree next to our classroom, small pieces of chalk, and my coffee cup.\\
One good demonstration consists of repeatedly throwing an object vertically up in the air and then catching it when it returns to your hand. Point out that when you first throw it up, it has an upward velocity. As it moves up, the velocity becomes less. At the top of its trajectory, it momentarily has a zero velocity. After that, it gains a downward velocity, at first a small one and then increasing in magnitude.\\
If you are walking across the classroom at a constant rate while performing this demonstration, you can additionally show that the projectile continues to move horizontally at the same rate, matching your motion. This shows that the horizontal velocity of a projectile is a constant.}
\end{itemize}

%Speed, Velocity, Acceleration


%Equations of Motion


%Motion Under Gravity / Determination of Acceleration Due to Gravity (pendulum)

\subsection{Simple Oscillator}
\begin{itemize}
\item{Preparation time: 1 minute}
\item{Materials: Spring, thread or piece of rubber strip, several weights}
\item{Procedure: Attach the spring or rubber strip to your weight. The weight could be anything: a laboratory weight, a set of keys, or a small padlock. Start the weight oscillating, while explaining to the students how simple harmonic motion works. Add more weight (more keys, another padlock) and observe that there is no change to the period. Now increase or decrease the length of the pendulum and observe any changes to the period. You can tabulate the results for different masses and lengths (keeping one thing constant each time) so that students can see experimentally the dependence of period on length or mass.}
\item{Theory: The period of a pendulum depends on the length of the pendulum (neglecting air resistance), so no change should be noticed if the mass is changed.}
\end{itemize}

